\documentclass[utf8x]{beamer}

\usepackage[utf8x]{inputenc}
\usepackage[OT1]{fontenc}
\usepackage{graphicx}
\usepackage{amsmath}
\usepackage{listings}
\usepackage{hyperref}
\usepackage{xcolor}
%\logo{\includegraphics[width=.8in]{UdeM_NoirBleu_logo_Marie_crop}}


\usetheme{Malmoe}  % Now it's a beamer presentation with the lisa theme!
\usecolortheme{beaver}
\setbeamertemplate{footline}[page number]
\setbeamertemplate{navigation symbols}{}

\lstloadlanguages{Python,C,sh}

\definecolor{darkgreen}{RGB}{0,93,21}
\definecolor{greenblue}{RGB}{40,110,126}
\definecolor{lightgray}{RGB}{246,246,246}
\definecolor{bordergray}{RGB}{193,193,193}
\definecolor{lightblue}{RGB}{0,114,168}
\definecolor{methblue}{RGB}{0,31,108}

\newcommand{\superscript}[1]{\ensuremath{^{\textrm{#1}}}}

\mode<presentation>

\title{Introduction to Theano}
\author{%
\footnotesize
Arnaud Bergeron \newline
(slides adapted by Frédéric Bastien from slides by Ian G.) \newline
(further adapted by Arnaud Bergeron)
}
\date{February 26, 2015}

\lstset{
language=Python,
basicstyle=\fontfamily{pcr}\selectfont\footnotesize,
keywordstyle=\color{darkgreen}\bfseries,
commentstyle=\color{greenblue}\itshape,
%commentstyle=\color{blue}\itshape,
stringstyle=\color{violet},
showstringspaces=false,
tabsize=4,
backgroundcolor=\color{lightgray},
frame=single,
emph={[2]__init__,make_node,perform,infer_shape,c_code,make_thunk,grad,R_op},emphstyle={[2]\color{methblue}},
emph={[3]self},emphstyle={[3]\color{darkgreen}},
moredelim=**[is][{\color{red}}]{`}{`}
}

\newcommand{\code}[1]{\lstinline[emph={[2]}]|#1|}

\begin{document}

\begin{frame}[plain]
 \titlepage
% \vspace{-5em}
% \includegraphics[width=1in]{../hpcs2011_tutorial/pics/lisabook_logo_text_3.png}
% \hfill
% \includegraphics[width=.8in]{../hpcs2011_tutorial/pics/UdeM_NoirBleu_logo_Marie_crop}
\end{frame}

\section{Outline}
\begin{frame}{High level}\setcounter{page}{1}
  \begin{itemize}
  \item Overview of library (3 min)
  \item Building expressions (30 min)
  \item Compiling and running expressions (30 min)
  \item Modifying expressions (25 min)
  \item Debugging (30 min)
  \item Citing Theano (2 min)
  \end{itemize}
\end{frame}


\begin{frame}{Overview of Library}
  Theano is many things
  \begin{itemize}
  \item Language
  \item Compiler
  \item Python library
  \end{itemize}
\end{frame}

\begin{frame}{Overview}
  Theano language:
  \begin{itemize}
  \item Operations on scalar, vector, matrix, tensor, and sparse variables
  \item Linear algebra
  \item Element-wise nonlinearities
  \item Convolution
  \item Extensible
  \end{itemize}
\end{frame}

\begin{frame}[fragile]
  \frametitle{Overview}
  Using Theano:
  \begin{itemize}
  \item define expression $f(x,y) = x + y$
  \item compile expression
\lstset{language=Python,
        commentstyle=\itshape\color{blue},
        stringstyle=\color{violet},
        }
\begin{lstlisting}
int f(int x, int y){
  return x + y;
}
\end{lstlisting}

  \item execute expression
\lstset{language=Python,
        commentstyle=\itshape\color{blue},
        stringstyle=\color{violet},
        }
\begin{lstlisting}
>>> f(1, 2)
3
\end{lstlisting}
  \end{itemize}
\end{frame}


\section{Building}
\begin{frame}{Building expressions}
  \begin{itemize}
  \item Scalars
  \item Vectors
  \item Matrices
  \item Tensors
  \item Reduction
  \item Dimshuffle
  \end{itemize}
\end{frame}

\begin{frame}[fragile]
  \frametitle{Scalar math}
  Using Theano:
  \begin{itemize}
  \item define expression $f(x,y) = x + y$
  \item compile expression
  \end{itemize}
\lstset{language=Python,
        commentstyle=\itshape\color{blue},
        stringstyle=\color{violet},
        }
\begin{lstlisting}
from theano import tensor as T
x = T.scalar()
y = T.scalar()
z = x+y
w = z*x
a = T.sqrt(w)
b = T.exp(a)
c = a ** b
d = T.log(c)
\end{lstlisting}
\end{frame}

\begin{frame}[fragile]
  \frametitle{Vector math}

\lstset{language=Python,
        commentstyle=\itshape\color{blue},
        stringstyle=\color{violet},
        }
\begin{lstlisting}
from theano import tensor as T
x = T.vector()
y = T.vector()
# Scalar math applied elementwise
a = x * y
# Vector dot product
b = T.dot(x, y)
# Broadcasting
c = a + b
\end{lstlisting}
\end{frame}

\begin{frame}[fragile]
  \frametitle{Matrix math}

\lstset{language=Python,
        commentstyle=\itshape\color{blue},
        stringstyle=\color{violet},
        }
\begin{lstlisting}
from theano import tensor as T
x = T.matrix()
y = T.matrix()
a = T.vector()
# Matrix-matrix product
b = T.dot(x, y)
# Matrix-vector product
c = T.dot(x, a)
\end{lstlisting}
\end{frame}

\begin{frame}[fragile]
  \frametitle{Tensors}
  Using Theano:
  \begin{itemize}
  \item define expression $f(x,y) = x + y$
  \item compile expression
    \begin{itemize}
    \item Dimensionality defined by length of ``broadcastable'' argument
    \item Can add (or do other elemwise op) on two
      tensors with same dimensionality
    \item Duplicate tensors along broadcastable axes to
      make size match
    \end{itemize}
  \end{itemize}
\lstset{language=Python,
        commentstyle=\itshape\color{blue},
        stringstyle=\color{violet},
        }
\begin{lstlisting}
from theano import tensor as T
tensor3 = T.TensorType(
    broadcastable=(False, False, False),
    dtype='float32')
x = tensor3()
\end{lstlisting}
\end{frame}

\begin{frame}[fragile]
  \frametitle{Reductions}
  Using Theano:
  \begin{itemize}
  \item define expression $f(x,y) = x + y$
  \item compile expression
  \end{itemize}
\lstset{language=Python,
        commentstyle=\itshape\color{blue},
        stringstyle=\color{violet},
        }
\begin{lstlisting}
from theano import tensor as T
tensor3 = T.TensorType(
    broadcastable=(False, False, False),
    dtype='float32')
x = tensor3()
total = x.sum()
marginals = x.sum(axis=(0, 2))
mx = x.max(axis=1)
\end{lstlisting}
\end{frame}

\begin{frame}[fragile]
  \frametitle{Dimshuffle}

\lstset{language=Python,
        commentstyle=\itshape\color{blue},
        stringstyle=\color{violet},
        }
\begin{lstlisting}
from theano import tensor as T
tensor3 = T.TensorType(broadcastable=(False, False, False), dtype=''float32'')
x = tensor3()
y = x.dimshuffle((2, 1, 0))
a = T.matrix()
b = a.T
# Same as b
c = a.dimshuffle((0, 1))
# Adding to larger tensor
d = a.dimshuffle((0, 1, ``x''))
e = a + d
\end{lstlisting}
\end{frame}

\begin{frame}{Exercices}
  Work through the ``01\_buildbing\_expressions'' directory now.
  Available at ``git~clone~https://github.com/nouiz/ccw\_tutorial\_theano.git''.
\end{frame}

\section{Compiling/Running}
\begin{frame}{Compiling and running expression}
  \begin{itemize}
  \item theano.function
  \item shared variables and updates
  \item compilation modes
  \item compilation for GPU
  \item optimizations
  \end{itemize}
\end{frame}

\begin{frame}[fragile]
  \frametitle{theano.function}

\lstset{language=Python,
        commentstyle=\itshape\color{blue},
        stringstyle=\color{violet},
        }
\begin{lstlisting}
>>> from theano import tensor as T
>>> x = T.scalar()
>>> y = T.scalar()
>>> from theano import function
>>> # first arg is list of SYMBOLIC inputs
>>> # second arg is SYMBOLIC output
>>> f = function([x, y], x + y)
>>> # Call it with NUMERICAL values
>>> # Get a NUMERICAL output
>>> f(1., 2.)
array(3.0)
\end{lstlisting}
\end{frame}

\begin{frame}{Shared variables}
  \begin{itemize}
  \item It’s hard to do much with purely functional programming
  \item ``shared variables'' add just a little bit of imperative programming
  \item A “shared variable” is a buffer that stores a numerical value for a Theano variable
  \item Can write to as many shared variables as you want, once each, at the end of the function
  \item  Modify outside Theano function with get\_value() and set\_value() methods.
  \end{itemize}
\end{frame}

\begin{frame}[fragile]
  \frametitle{Shared variable example}

\lstset{language=Python,
        commentstyle=\itshape\color{blue},
        stringstyle=\color{violet},
        }
\begin{lstlisting}
>>> from theano import shared
>>> x = shared(0.)
>>> from theano.compat.python2x import OrderedDict
>>> updates = OrderedDict()
>>> updates[x] = x + 1
>>> f = function([], updates=updates)
>>> f()
>>> x.get\_value()
1.0
>>> x.set\_value(100.)
>>> f()
>>> x.get\_value()
101.0
\end{lstlisting}
\end{frame}

\begin{frame}{Which dict?}
  \begin{itemize}
  \item Use theano.compat.python2x.OrderedDict
  \item Not collections.OrderedDict
  \begin{itemize}
  \item This isn’t available in older versions of python,
and will limit the portability of your code
  \end{itemize}
  \item Not \{\} aka dict
  \begin{itemize}
  \item The iteration order of this built-in class is not
    deterministic (thanks, Python!) so if Theano
    accepted this, the same script could compile
    different C programs each time you run it
  \end{itemize}
  \end{itemize}
\end{frame}

\begin{frame}{Compilation modes}
  \begin{itemize}
  \item Can compile in different modes to get different kinds of programs
  \item Can specify these modes very precisely with arguments to theano.function
  \item Can use a few quick presets with environment variable flags
  \end{itemize}
\end{frame}

\begin{frame}{Example preset compilation modes}
  \begin{itemize}
  \item FAST\_RUN: default. Spends a lot of time on
compilation to get an executable that runs
fast.
  \item FAST\_COMPILE: Doesn’t spend much time
compiling. Executable usually uses python
instead of compiled C code. Runs slow.
  \item DEBUG\_MODE: Adds lots of checks.
Raises error messages in situations other
modes regard as fine.
  \end{itemize}
\end{frame}

\begin{frame}{Compilation for GPU}
  \begin{itemize}
  \item Theano current back-end only supports 32 bit on GPU
  \item CUDA supports 64 bit, but is slow in gamer card
  \item T.fscalar, T.fvector, T.fmatrix are all 32 bit
  \item T.scalar, T.vector, T.matrix resolve to 32 bit or 64 bit depending on theano’s floatX flag
  \item floatX is float64 by default, set it to float32
  \item Set device flag to gpu (or a specific gpu, like gpu0)
  \end{itemize}
\end{frame}

\begin{frame}{Optimizations}
  \begin{itemize}
  \item Theano changes the symbolic expressions
    you write before converting them to C code
  \item It makes them faster
  \begin{itemize}
  \item (x+y)+(x+y) -> 2 (x + y)
  \end{itemize}
  \item It makes them more stable
  \begin{itemize}
  \item exp(a)/exp(a).sum()->softmax(a)
  \end{itemize}
  \end{itemize}
\end{frame}

\begin{frame}[fragile]
  \frametitle{Optimizations}

  \begin{itemize}
  \item Sometimes optimizations discard error
    checking and produce incorrect output
    rather than an exception
  \end{itemize}
\lstset{language=Python,
        commentstyle=\itshape\color{blue},
        stringstyle=\color{violet},
        }
\begin{lstlisting}
>>> x = T.scalar()
>>> f = function([x], x/x)
>>> f(0.)
array(1.0)
\end{lstlisting}

\end{frame}

\begin{frame}{Exercises}
Work through the ``02\_compiling\_and\_running'' directory now
\end{frame}

\section{Modifying expressions}
\begin{frame}{Modifying expressions}
  \begin{itemize}
  \item The grad method
  \item Variable nodes
  \item Types
  \item Ops
  \item Apply nodes
  \end{itemize}
\end{frame}

\begin{frame}[fragile]
  \frametitle{The grad method}

\lstset{language=Python,
        commentstyle=\itshape\color{blue},
        stringstyle=\color{violet},
        }
\begin{lstlisting}
>>> x = T.scalar('x')
>>> y = 2. * x
>>> g = T.grad(y, x)
>>> from theano.printing import min_informative_str
>>> print min_informative_str(g)
A. Elemwise{mul}
 B. Elemwise{second,no_inplace}
  C. Elemwise{mul,no_inplace}
   D. TensorConstant{2.0}
   E. x
  F. TensorConstant{1.0}
 <D>
\end{lstlisting}
\end{frame}

\begin{frame}{Theano Variables}
  \begin{itemize}
  \item A Variable is a theano expression
  \item Can come from T.scalar, T.matrix, etc.
  \item Can come from doing operations on other Variables
  \item Every Variable has a type field, identifying its Type \newline
    e.g. TensorType((True, False), ‘float32’)
  \item Variables can be thought of as nodes in a graph
  \end{itemize}
\end{frame}

\begin{frame}{Ops}

  \begin{itemize}
  \item  An Op is any class that describes a
mathematical function of some variables
  \item Can call the op on some variables to get a
new variable or variables
  \item An Op class can supply other forms of
information about the function, such as its
derivatives
  \end{itemize}
\end{frame}

\begin{frame}{Apply nodes}
  \begin{itemize}
  \item The Apply class is a specific instance of an application of an Op
  \item Notable fields:
    \begin{itemize}
    \item op: The Op to be applied
    \item inputs: The Variables to be used as input
    \item outputs: The Variables produced
    \end{itemize}
  \item Variable.owner identifies the Apply that created the variable
  \item Variable and Apply instances are nodes and owner/
    inputs/outputs identify edges in a Theano graph
  \end{itemize}
\end{frame}

\begin{frame}{Exercises}
Work through the ``03\_modifying'' directory now
\end{frame}

\section{Debugging}
\begin{frame}{Debugging}
  \begin{itemize}
  \item DEBUG\_MODE
  \item Error message
  \item theano.printing.debugprint
  \item min\_informative\_str
  \item compute\_test\_value
  \item Accessing the FunctionGraph
  \end{itemize}
\end{frame}

\begin{frame}[fragile]
  \frametitle{Error message: code}
\lstset{language=Python,
        commentstyle=\itshape\color{blue},
        stringstyle=\color{violet},
        }
\begin{lstlisting}
import numpy as np
import theano
import theano.tensor as T
x = T.vector()
y = T.vector()
z = x + x
z = z + y
f = theano.function([x, y], z)
f(np.ones((2,)), np.ones((3,)))
\end{lstlisting}
\end{frame}

\begin{frame}[fragile]
  \frametitle{Error message: 1st part}

\lstset{language=Python,
        commentstyle=\itshape\color{blue},
        stringstyle=\color{violet},
        }
\begin{lstlisting}
Traceback (most recent call last):
[...]
ValueError: Input dimension mis-match.
    (input[0].shape[0] = 3, input[1].shape[0] = 2)
Apply node that caused the error:
   Elemwise{add,no_inplace}(<TensorType(float64, vector)>,
                            <TensorType(float64, vector)>,
                            <TensorType(float64, vector)>)
Inputs types: [TensorType(float64, vector),
               TensorType(float64, vector),
               TensorType(float64, vector)]
Inputs shapes: [(3,), (2,), (2,)]
Inputs strides: [(8,), (8,), (8,)]
Inputs scalar values: ['not scalar', 'not scalar', 'not scalar']
\end{lstlisting}
\end{frame}

\begin{frame}[fragile]
  \frametitle{Error message: 2st part}

\lstset{language=Python,
        commentstyle=\itshape\color{blue},
        stringstyle=\color{violet},
        }
\begin{lstlisting}
HINT: Re-running with most Theano optimization
disabled could give you a back-traces when this
node was created. This can be done with by setting
the Theano flags optimizer=fast_compile
HINT: Use the Theano flag 'exception_verbosity=high'
for a debugprint of this apply node.
\end{lstlisting}
\end{frame}

\begin{frame}[fragile]
  \frametitle{Error message: exception\_verbosity=high}

\lstset{language=Python,
        commentstyle=\itshape\color{blue},
        stringstyle=\color{violet},
        }
\begin{lstlisting}
Debugprint of the apply node:
Elemwise{add,no_inplace} [@A] <TensorType(float64, vector)> ''
 |<TensorType(float64, vector)> [@B] <TensorType(float64, vector)>
 |<TensorType(float64, vector)> [@C] <TensorType(float64, vector)>
 |<TensorType(float64, vector)> [@C] <TensorType(float64, vector)>

\end{lstlisting}
\end{frame}

\begin{frame}[fragile]
  \frametitle{Error message: optimizer=fast\_compile}

\lstset{language=Python,
        commentstyle=\itshape\color{blue},
        stringstyle=\color{violet},
        }
\begin{lstlisting}
Backtrace when the node is created:
  File "test.py", line 7, in <module>
    z = z + y
  File "/home/nouiz/src/Theano/theano/tensor/var.py", line 122, in __add__
    return theano.tensor.basic.add(self, other)

\end{lstlisting}
\end{frame}

\begin{frame}[fragile]
  \frametitle{Error message: Traceback}

\lstset{language=Python,
        commentstyle=\itshape\color{blue},
        stringstyle=\color{violet},
        }
\begin{lstlisting}
Traceback (most recent call last):
  File "test.py", line 9, in <module>
    f(np.ones((2,)), np.ones((3,)))
  File "/u/bastienf/repos/theano/compile/function_module.py",
       line 589, in __call__
    self.fn.thunks[self.fn.position_of_error])
  File "/u/bastienf/repos/theano/compile/function_module.py",
       line 579, in __call__
    outputs = self.fn()

\end{lstlisting}
\end{frame}

\begin{frame}[fragile]
  \frametitle{debugprint}

\lstset{language=Python,
        commentstyle=\itshape\color{blue},
        stringstyle=\color{violet},
        }
\begin{lstlisting}
>>> from theano.printing import debugprint
>>> debugprint(a)
Elemwise{mul,no_inplace} [@A] ''
 |TensorConstant{2.0} [@B]
 |Elemwise{add,no_inplace} [@C] 'z'
   |<TensorType(float64, scalar)> [@D]
   |<TensorType(float64, scalar)> [@E]
\end{lstlisting}
\end{frame}

\begin{frame}[fragile]
  \frametitle{min\_informative\_str}

\lstset{language=Python,
        commentstyle=\itshape\color{blue},
        stringstyle=\color{violet},
        }
\begin{lstlisting}
>>> x = T.scalar()
>>> y = T.scalar()
>>> z = x + y
>>> z.name = 'z'
>>> a = 2. * z
>>> from theano.printing import min_informative_str
>>> print min_informative_str(a)
A. Elemwise{mul,no_inplace}
 B. TensorConstant{2.0}
 C. z
\end{lstlisting}
\end{frame}

\begin{frame}[fragile]
  \frametitle{compute\_test\_value}

\lstset{language=Python,
        commentstyle=\itshape\color{blue},
        stringstyle=\color{violet},
        }
\begin{lstlisting}
>>> from theano import config
>>> config.compute_test_value = 'raise'
>>> x = T.vector()
>>> import numpy as np
>>> x.tag.test_value = np.ones((2,))
>>> y = T.vector()
>>> y.tag.test_value = np.ones((3,))
>>> x + y
...
ValueError: Input dimension mis-match.
(input[0].shape[0] = 2, input[1].shape[0] = 3)
\end{lstlisting}
\end{frame}

\begin{frame}[fragile]
  \frametitle{Accessing a function’s fgraph}

\lstset{language=Python,
        commentstyle=\itshape\color{blue},
        stringstyle=\color{violet},
        }
\begin{lstlisting}
>>> x = T.scalar()
>>> y = x / x
>>> f = function([x], y)
>>> debugprint(f.maker.fgraph.outputs[0])
DeepCopyOp [@A] ''
 |TensorConstant{1.0} [@B]
\end{lstlisting}
\end{frame}

\begin{frame}{Exercises}
Work through the ``04\_debugging'' directory now
\end{frame}

\section{Citing}
\begin{frame}{Citing Theano}
  \begin{itemize}
  \item Please cite both of the following papers in
all work that uses Theano:
  \item Bastien, Frédéric, Lamblin, Pascal, Pascanu, Razvan, Bergstra, James, Goodfellow, Ian, Bergeron, Arnaud, Bouchard, Nicolas, and
     Bengio,Yoshua. Theano: new features and speed improvements. Deep Learning and Unsupervised Feature Learning NIPS 2012
    Workshop, 2012.
  \item Bergstra, James, Breuleux, Olivier, Bastien, Frédéric, Lamblin, Pascal, Pascanu, Razvan, Desjardins, Guillaume, Turian, Joseph, Warde-
     Farley, David, and Bengio,Yoshua. Theano: a CPU and GPU math expression compiler. In Proceedings of the Python for Scientific
      Computing Conference (SciPy), June 2010. Oral Presentation.
  \end{itemize}
\end{frame}

\begin{frame}{Example acknowledgments}
We would like to thank the developers of
Theano \\citep\{bergstra+al:2010-scipy,Bastien-Theano-2012\},
Pylearn2 \\citep\{pylearn2\_arxiv\_2013\}. We would also like
to thank NSERC, Compute Canada, and Calcul Qu\'ebec
for providing computational resources.
\end{frame}


\begin{frame}
\begin{center}
\bibliography{strings,strings-short,ml,aigaion-shorter}
\Huge
Questions?
\end{center}
\end{frame}


\end{document}
